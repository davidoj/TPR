%!TEX root = main.tex

\subsection{Key Questions}

Key questions arising from this literature search include:

\begin{itemize}
    \item The decision rule in statistical decision problems maps observed data to a decision. However, in this setting the decision has no consequences. Causal decision theories consider which actions should be taken in light of their consequences, but observed data has no role in these theories. What is the appropriate general description of a statistical causal decision problem - that is, a problem where the objective is to select a loss-minimising decision function where decisions may have consequences.
    \item In my view, there are advantages to formulating problems in terms of consequences and losses over formulating them in terms of counterfactual effects. Is it possible to map decision rules implicit in a counterfactual effects formulation to a consequences and losses formulation in cases where the counterfactual effects can be estimated?
    \item The structural risk minimisation principle balances empirical and generalisation loss. Causal decision problems introduce the ``loss due to erroneous consequences'' which can be large even when empirical and generalisation losses are zero. Some approaches to causal discovery employ principles such as MDL that are backed by theoretical bounds on generalisation error. Is there any connection that can be made between generalisation loss and erroneous consequence loss?
    \item Given a causal inference problem in the observational setting, what is the relationship between structural intervention distance and erroneous consequence loss?
    \item Given an oracle for certain elements of a causal structure, how do loss bounds depend on which parts of the structure is revealed?
\end{itemize}